\begin{savequote}[45mm]
\ascii{Any fool can write code that a computer can understand. Good programmers write code that humans can understand.}
\qauthor{\ascii{- Martin Flower}}
\end{savequote}

\chapter{Design Priciples} 
\label{ch:design-priciples}

\section{SOLID}

\begin{content}

\ascii{SOLID}完整地描述是由\ascii{Robert C. Martin}在其脍炙人口的著作《敏捷软件开发,原则、模式与实践》中阐述的,成为了软件设计最基本的原则。

\begin{principle}
\ascii{SRP(The Single Responsibility Principle)}
\end{principle}

\begin{enum}
  \eitem{\ascii{SRP, The Single Responsibility Principle}}
  \eitem{\ascii{OCP, The Open Closed Principle}}
  \eitem{\ascii{LSP, The Liskov Substitution Principle}}
  \eitem{\ascii{ISP, The Interface Segregation Principle}}
  \eitem{\ascii{DIP, The Dependency Inversion Principle}}
\end{enum}

\end{content}

\section{Simple Design Principles}

\begin{content}

这是\ascii{XP}倡导的最基本的四个简单设计原则,其重要性依次进行排列。

\begin{enum}
  \eitem{\ascii{runs all the tests}}
  \eitem{\ascii{says everything OnceAndOnlyOnce}}
  \eitem{\ascii{expresses every idea that we need to express, Self-documenting code}}
  \eitem{\ascii{has no superfluous parts}}
\end{enum}

\end{content}

\section{Orthotropic Design Principles}

\begin{content}

这是袁英杰在其\ascii{OO}训练营中阐述的理论和指导原则,并对作者产生了巨大的影响,正交设计四原则成为了作者在设计中最具有指导意义的原则之一。

\begin{enum}
  \eitem{消除重复}
  \eitem{分离关注点}
  \eitem{最小依赖}
  \eitem{向稳定的方向依赖}
\end{enum}

\end{content}

\section{Others}

\begin{content}

\begin{enum}
  \eitem{\ascii{DRY, Don't repeat youself}}
  \eitem{\ascii{Law of Demeter}}
  \eitem{\ascii{YAGNI, You ain't gonna need it}}
  \eitem{\ascii{KISS, Keep it simple, stupid}}
  \eitem{\ascii{CQS, Command-Query separation}}
  \eitem{\ascii{Hollywood Principle, Don't call us, we'll call you}}
\end{enum}

\end{content}
