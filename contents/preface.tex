\chapter{前言} 
\label{ch:preface}

\begin{content}

\ascii{C++ Programming Style}是一份关于\ascii{C++}的代码规范,为\ascii{C++}程序员提供程序设计的参考和建议。

规范总结了我们长期在编码实践中遇到的一些宝贵经验,其中部分规则、原则和建议摘自社区软件设计大师的论文、报告、书籍和博客,并引用了一些典型的源代码。

\end{content}

\section*{约定}

\begin{content}

\begin{table}[!htb]
\resizebox{0.95\textwidth}{!} {
\begin{tabular*}{1.2\textwidth}{@{}ll@{}}
\toprule
\ascii{约定} & \ascii{说明} \\
\midrule
\ascii{原则} & \ascii{坚持遵守的指导思想} \\
\ascii{规则} & \ascii{必须遵守的约定} \\ 
\ascii{建议} & \ascii{可以考虑的约定} \\ 
\ascii{正例} & \ascii{原则、规则、建议所给出的正确例子} \\ 
\ascii{反例} & \ascii{原则、规则、建议所给出的错误例子} \\ 
\bottomrule
\end{tabular*}
}
\caption{规范约定}
\label{tbl:regulation-tbl}
\end{table}

\end{content}

\section*{宏定义}

\begin{content}

为了提高代码的表现力,规范中使用了一部分实用的宏定义。为了避免歧义,这里列出了部分较为重要的宏定义供参考和查阅。

\begin{leftbar}
\begin{c++}[caption={\ttfamily{cub/base/Default.h}}]
#ifndef GKOQWPRT_1038935_NCVBNMZHJS_8909603
#define GKOQWPRT_1038935_NCVBNMZHJS_8909603

namespace details
{
   template <typename T>
   struct DefaultValue
   {
      static T value()
      {
         return T();
      }
   };

   template <typename T>
   struct DefaultValue<T*>
   {
       static T* value()
       {
           return 0;
       }
   };

   template <typename T>
   struct DefaultValue<const T*>
   {
       static T* value()
       {
           return 0;
       }
   };

   template <>
   struct DefaultValue<void>
   {
      static void value()
      {
      }
   };
}

#define DEFAULT(type, method)  \
    virtual type method { return ::details::DefaultValue<type>::value(); }

#endif
\end{c++}
\end{leftbar}

\ascii{DEFAULT}对于定义空实现的\ascii{virtual}函数非常方便。需要注意的是,\ascii{DEFAULT}所有计算都是发生在编译时,并没有任何运行时成本。

\begin{leftbar}
\begin{c++}[caption={\ttfamily{cub/base/Keywords.h}}]
#ifndef H16274882_9153_4DB2_A2E2_F23D4CCB9381
#define H16274882_9153_4DB2_A2E2_F23D4CCB9381

#include <cub/base/Config.h>
#include <cub/base/Default.h>

#define ABSTRACT(...) virtual __VA_ARGS__ = 0

#if __SUPPORT_VIRTUAL_OVERRIDE
#   define OVERRIDE(...) virtual __VA_ARGS__ override
#else
#   define OVERRIDE(...) virtual __VA_ARGS__
#endif

#define EXTENDS(...) , ##__VA_ARGS__
#define IMPLEMENTS(...) EXTENDS(__VA_ARGS__)

#endif
\end{c++}
\end{leftbar}

\ascii{Config.h}提供了编译器支持\ascii{C++11}特性的配置信息。\ascii{ABSTRACT, OVERRIDE, EXTENDS, IMPLEMENTS}等关键字,使得其他阵营的程序员,也能看懂规范中大部分\ascii{C++}的代码,极大地改善了\ascii{C++}的表现力。

\begin{leftbar}
\begin{c++}[caption={\ttfamily{cub/base/Role.h}}]
#ifndef HF95EF112_D6C6_4DB0_8C1A_BE5A6CF8E3F1
#define HF95EF112_D6C6_4DB0_8C1A_BE5A6CF8E3F1

#include <cub/base/Keywords.h>

namespace details
{
   template <typename T>
   struct Role
   {
      virtual ~Role() {}
   };
}

#define DEFINE_ROLE(type) struct type : ::details::Role<type>

#endif
\end{c++}
\end{leftbar}

通过\ascii{DEFINE\_ROLE}的宏定义来实现对接口的定义,可以消除子类对虚拟析构函数的重复实现,消除代码重复。

\end{content}

