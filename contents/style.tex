\begin{savequote}[45mm]
\ascii{Do the simplest thing that could possibly work.}
\qauthor{\ascii{- Kent Beck}}
\end{savequote}

\chapter{代码风格} 
\label{ch:coding-style}

\section{开发环境}

\begin{content}

\begin{regulation}
系统中所有的代码看起来就好像是由单独一个值得胜任的人编写的,团队遵循统一的,一致的,没有歧义的代码规范。\footnote{\ascii{Agile Software Development, Principles, Patterns, and Practices, Robert C. Martin}}。
\end{regulation}

我们倡导\ascii{XP}的价值观,致力于优秀的软件设计。遵循统一的代码规范,可以使的团队内部形成统一的代码风格,降低沟通成本,增强系统的可理解性。

\begin{regulation}
为了保证代码的最大可移植性,团队使用的操作系统、编译器类型、版本等基础设施应保持一致性。
\end{regulation}

不推荐在\ascii{Windows}上开发\ascii{C/C++}程序;推荐使用\ascii{Linux},\ascii{MacOS}等类\ascii{Unix}操作系统。

\begin{regulation}
团队统一使用相同的集成开发环境\ascii{(IDE)},并统一\ascii{IDE}的主题风格,使用相同的代码模板,保持团队内部代码风格的一致性。
\end{regulation}

推荐两款集成开发环境\ascii{(IDE)},因为它们都支持良好的重构功能:
\begin{enum}
  \eitem{\ascii{Eclipse}}
  \eitem{\ascii{Clion}}
\end{enum}

\begin{regulation}
团队统一配置\ascii{IDE}使用等宽字体
\end{regulation}

如果团队中有人使用非等宽字体,会造成团队内部代码对不齐、缩进混乱的情况。

\begin{regulation}
团队统一配置\ascii{IDE}的文件编码格式,避免中文出现乱码的现象
\end{regulation}

强烈推荐遵循良好的代码注释的规范和最佳实践;倡导整洁代码,代码自解释;即使在特殊场景需要注释时,应使用标准的英文注释。

\begin{regulation}
团队统一配置\ascii{IDE}的\ascii{TAB}为相同数目的空格,坚决抵制使用\ascii{TAB}对齐代码。
\end{regulation}

因各种编辑器解释\ascii{TAB}的长度存在不一致,如果使用\ascii{TAB}对齐代码,可能会使代码缩进混乱不堪。\footnote{\ascii{TAB}一般配置为2或4个空格,所有的编辑器都提供了类似的配置接口。}

\end{content}

\section{代码风格}

\begin{content}

\begin{regulation}
团队应该保持一致的命名、缩进、空格、空行、断行的代码风格。
\end{regulation}

业界存在多种经典的代码风格,各自都拥有独特的优势,团队应该选择并保持其中一种代码风格。\footnote{本规范采用的是\ascii{BSD/Allman}的代码风格。}

\begin{enum}
  \eitem{\ascii{K\&R}}
  \eitem{\ascii{BSD/Allman}}
  \eitem{\ascii{GNU}}
  \eitem{\ascii{Whitesmiths}}
\end{enum}

统一代码风格并非难事,团队发布统一的\ascii{IDE}代码模板,及其定制一个方便的代码格式化的快捷键,即可解决所有对齐、缩进、空格、断行等问题。

\begin{regulation}
程序实体之间有且仅有一行空行。
\end{regulation}

函数之间的空行,能够帮组我们快速定位函数的始末的准确位置;甚至在函数内部,将逻辑相关的代码放在一起也同样具有意义(理论应该提取该函数,并合理命名),它能够帮助我们更好地理解代码块的语义。

超过一行的空行完全没有必要,部分粗心的程序员在处理这些细节时总存在着或多或少的问题,团队应该杜绝这样的情况发生。

\begin{regulation}
每个文件末尾都应该有且仅有一行空行。
\end{regulation}

如果使用\ascii{Eclipse},在创建新文件时自动地在文件末添加一行空行;但诸如\ascii{Visual }\cpp{}等情况下,则需要程序员在文件末自行手动地添加一行空行。这样做可以最大化地实现代码的可移植性,避免某些编译器因缺少空行而产生警告信息。

\begin{regulation}
每一行的代码或注释不得超过\ascii{80}列;否则将其拆分为若干行,并保持适当缩进,保证排版整洁漂亮。
\end{regulation}

长表达式、长语句或多或少都存在重复,此时往往是函数提取、概念整合的最佳时机,缩短长表达式、长语句可以极大地改善代码的可读性,缩短代码阅读和理解的时间。

\end{content}
